% ==========================================
% BAB II STUDI LITERATUR
% ==========================================
\chapter{STUDI LITERATUR}
\label{chap:studi-literatur}
\section{Perguruan Tinggi dan Pemeringkatan}
Perkembangan pendidikan tinggi dalam dua dekade terakhir tidak dapat dilepaskan dari kemunculan berbagai sistem pemeringkatan global, seperti Academic Ranking of World Universities (ARWU), QS World University Rankings (QS WUR), dan Times Higher Education (THE). Pemeringkatan perguruan tinggi kini telah menjadi salah satu instrumen yang digunakan secara luas oleh berbagai pemangku kepentingan mulai dari pemerintah, calon mahasiswa, hingga lembaga pendanaan untuk menilai reputasi dan daya saing institusi pendidikan tinggi.

Pemeringkatan telah menjadi salah satu kekuatan eksternal yang paling berpengaruh dalam membentuk kebijakan dan perilaku perguruan tinggi di seluruh dunia, tidak hanya berfungsi sebagai alat informasi, tetapi juga sebagai pendorong utama dalam perencanaan strategis, alokasi sumber daya, dan pengambilan keputusan di tingkat institusi. Sebagian besar perguruan tinggi di Eropa telah memasukkan indikator pemeringkatan ke dalam rencana strategis mereka dan menggunakannya untuk tujuan benchmarking, komunikasi reputasi, serta penyusunan prioritas kelembagaan \cite{hazelkornRankingsInstitutionalStrategies2014}.

Pemeringkatan memengaruhi keputusan institusi dalam berbagai aspek, termasuk perencanaan strategis, rekrutmen staf akademik, alokasi dana penelitian, dan strategi pemasaran institusi. Studi \textcite{velizInfluenceGlobalRankings2022} menunjukkan bahwa pemeringkatan global secara langsung memengaruhi proses perencanaan strategis, penentuan prioritas riset, serta kebijakan rekrutmen staf internasional, meskipun terdapat ketegangan antara misi sosial universitas dan tuntutan indikator pemeringkatan yang cenderung menekankan output riset berbasis sitasi internasional.

Di antara berbagai sistem pemeringkatan yang ada, QS World University Rankings merupakan salah satu yang paling banyak dirujuk oleh perguruan tinggi. Metodologi QS WUR menilai universitas berdasarkan kombinasi indikator. Perguruan tinggi memanfaatkan data QS untuk mengidentifikasi kekuatan dan kelemahan relatif terhadap kompetitor, merancang strategi internasionalisasi, serta mengomunikasikan reputasi kepada pemangku kepentingan global.

Dalam konteks Indonesia, pemeringkatan internasional khususnya QS WUR telah menjadi bagian dari wacana "world class university" dan kebijakan peningkatan daya saing perguruan tinggi, dengan pemerintah secara eksplisit mendorong perguruan tinggi untuk memperkuat penelitian, inovasi, dan kolaborasi internasional guna meningkatkan posisi dalam pemeringkatan global

Banyak perguruan tinggi kini memilih untuk memanfaatkan pemeringkatan secara lebih strategis dan kritis sebagai salah satu sumber informasi tambahan untuk mengelola kinerja institusi, meningkatkan transparansi data, serta memperkuat tata kelola berbasis bukti. Dalam konteks inilah, pemahaman yang baik mengenai indikator pemeringkatan khususnya QS WUR dan kemampuan memantau capaian indikator tersebut di tingkat institusi menjadi semakin penting bagi perguruan tinggi yang ingin mengelola kinerjanya secara terukur dan berkelanjutan.
\section{Visualisasi Data}
Visualisasi data merujuk pada teknik penyajian informasi dengan cara mengubahnya menjadi objek visual, seperti titik, garis, atau batang, yang disusun dalam bentuk grafik untuk memudahkan pemahaman \cite{salinasPrimerDataVisualization2020}. \textcite{noahiliinskyDesigningDataVisualizations2011} menegaskan bahwa visualisasi data adalah media komunikasi informasi yang efisien dan efektif, memungkinkan penyajian sejumlah besar data dengan cara yang mudah dipahami oleh pengguna. Dengan pendekatan ini, data kompleks diubah menjadi representasi visual sehingga informasi utama dapat diserap lebih cepat. Pendekatan ini berperan penting dalam menghubungkan pengumpulan data dengan interpretasi yang lebih bermakna. Secara keseluruhan, visualisasi data menjadi alat penting untuk \textit{decision support system}, karena kemampuan utamanya adalah mengubah data menjadi wawasan yang \textit{actionable}, mendukung proses pengambilan keputusan yang lebih cepat dan tepat.

\section{Teknologi dan Infrastruktur untuk Visualisasi Data}
Teknologi dan infrastruktur memegang peranan penting dalam mendukung proses visualisasi data yang efektif di lingkungan organisasi modern. Untuk menghasilkan visualisasi yang informatif, diperlukan lingkungan teknologi yang mampu menangani penyimpanan, pengolahan, dan penyajian data dalam skala besar dengan kecepatan tinggi. Komponen utama dalam infrastruktur ini meliputi \textit{cloud computing} sebagai fondasi penyimpanan dan komputasi, serta berbagai \textit{tools} visualisasi data yang memungkinkan transformasi data mentah menjadi informasi visual yang mudah dipahami. Setiap komponen memiliki peran spesifik dalam memastikan bahwa data dapat diakses, diolah, dan disajikan secara efisien dan akurat untuk mendukung pengambilan keputusan berbasis bukti.

\subsection{\textit{Cloud Computing}}
\textit{Cloud computing} merupakan model yang menyediakan akses sesuai permintaan melalui jaringan ke sekumpulan sumber daya komputasi yang dapat dikonfigurasi, seperti jaringan, server, penyimpanan, aplikasi, dan layanan, yang dapat dengan cepat disediakan serta dilepaskan dengan upaya manajemen minimal \cite{mellNISTDefinitionCloud2011}. Dengan menggunakan \textit{cloud}, organisasi dapat memanfaatkan layanan penyimpanan yang terpusat dan terdistribusi melalui internet tanpa perlu berinvestasi pada infrastruktur fisik yang mahal \cite{satriniaAnalisisKeamananDan2022}. \textit{Cloud} memberikan kemudahan dalam hal skalabilitas, yang memungkinkan organisasi untuk menyesuaikan kapasitas penyimpanan atau pemrosesan data sesuai dengan kebutuhan yang berkembang \cite{bahonoImplementasiTeknologiCloud2025}. 

Dalam konteks visualisasi data, \textit{cloud} menyediakan platform yang memungkinkan integrasi berbagai sumber data secara otomatis. Pengguna dapat mengakses data mereka kapan saja dan dari mana saja tanpa terbatas oleh lokasi geografis atau perangkat keras tertentu, sehingga memberikan fleksibilitas yang lebih besar dalam proses pengambilan keputusan dan analisis \cite{satriniaAnalisisKeamananDan2022}. Keuntungan lain dari \textit{cloud} adalah kemudahan dalam kolaborasi dan berbagi informasi antara tim yang berbeda di dalam organisasi. Dengan fitur berbagi data dan \textit{dashboard} yang terintegrasi, \textit{cloud} memungkinkan pengguna untuk bekerja sama dalam menganalisis dan menyajikan data dalam bentuk visual yang mudah dipahami.

Keamanan juga menjadi prioritas utama dalam penyediaan layanan \textit{cloud} \cite{juliaAnalisisKinerjaBasis2024}. Dengan adanya mekanisme enkripsi, autentikasi multi-faktor, dan pengaturan hak akses yang ketat, data dapat terlindungi dari ancaman eksternal. Selain itu, \textit{cloud} juga mendukung analisis \textit{big data} yang memungkinkan pengguna untuk mengelola dan menganalisis data dalam volume yang sangat besar dengan lebih efisien \cite{marinescuCloudComputingTheory2023}. Dengan teknologi \textit{cloud}, organisasi tidak hanya dapat mengelola data dengan lebih baik, tetapi juga dapat memanfaatkan potensi data untuk meningkatkan efisiensi operasional dan mendukung inovasi.

\subsection{\textit{Tools} Visualisasi Data}
\textit{Tools} visualisasi data merupakan perangkat lunak yang dirancang untuk mengubah data mentah menjadi representasi visual yang informatif dan mudah dipahami. Pemilihan \textit{tools} yang tepat sangat bergantung pada kebutuhan spesifik organisasi, kompleksitas data yang dikelola, tingkat keahlian pengguna, serta ekosistem teknologi yang sudah ada. Berbagai \textit{tools} visualisasi menawarkan fitur yang beragam, mulai dari kemampuan integrasi data dari berbagai sumber, pembuatan grafik dan \textit{dashboard} interaktif, hingga kolaborasi dan berbagi informasi secara \textit{real-time}. Dalam konteks perguruan tinggi, \textit{tools} visualisasi data berperan penting untuk menyajikan indikator kinerja institusional, memantau capaian strategis, serta mendukung transparansi informasi bagi berbagai pemangku kepentingan.

\subsubsection{Tableau}
Tableau merupakan salah satu platform visualisasi data yang banyak digunakan karena kemampuannya dalam menghasilkan visualisasi interaktif dengan antarmuka yang intuitif. Pendekatan visual \textit{query} memungkinkan pengguna membuat berbagai jenis grafik dan \textit{dashboard} melalui teknik \textit{drag-and-drop} tanpa memerlukan keahlian pemrograman yang tinggi, sehingga cocok digunakan bahkan oleh mahasiswa atau analis pemula yang baru belajar literasi data \cite{battLearningTableauData2020}. Keunggulan lain yang dilihat oleh \textcite{battLearningTableauData2020} adalah kemudahan Tableau dalam menghubungkan data, menyusun beberapa \textit{worksheet} menjadi satu \textit{dashboard}, dan membangun \textit{story} yang menarasikan pola yang ditemukan dalam data, sehingga proses eksplorasi dan komunikasi hasil analisis menjadi lebih sistematis. Platform ini mendukung beragam jenis visualisasi mulai dari grafik sederhana seperti grafik batang dan garis, hingga tampilan yang lebih kompleks seperti peta geografis, \textit{heatmap}, \textit{treemap}, dan \textit{dashboard} multi-dimensi yang dapat disesuaikan dengan kebutuhan analisis.

Selain itu, Tableau juga menyediakan layanan Tableau \textit{Cloud}. Tableau Cloud adalah versi \textit{Software-as-a-Service} (SaaS) dari Tableau yang berjalan sepenuhnya di infrastruktur \textit{cloud} milik Tableau, sehingga organisasi tidak perlu mengelola server sendiri. Pengguna cukup memublikasikan \textit{dashboard} dari Tableau Desktop ke \textit{environment cloud} dan membagikannya melalui browser kepada pihak yang berkepentingan. Di Tableau Cloud, hak akses dapat diatur berdasarkan peran, \textit{dashboard} dapat dihubungkan ke berbagai sumber data \textit{cloud} seperti BigQuery, Snowflake, atau basis data lain yang terintegrasi, serta didukung fitur kolaborasi seperti berbagi tautan \textit{dashboard}, komentar, dan langganan email berkala. Secara arsitektur, seperti ditunjukkan pada \ref{gambar:arsitektur-tableau-cloud}, sumber data berada di sisi kiri (basis data \textit{on-premise} maupun \textit{cloud}), Tableau Cloud berada di tengah sebagai lapisan analitik dan visualisasi, sedangkan berbagai jenis klien di sisi kanan (browser di laptop, tablet, dan ponsel) mengakses \textit{dashboard} secara \textit{real-time} melalui internet.
\begin{figure}[H]
    \centering
    \includegraphics[width=0.7\textwidth]{image/architecture_tableau_cloud.png}
    \caption{Arsitektur Tableau Cloud}
    \label{gambar:arsitektur-tableau-cloud}
\end{figure}

\subsubsection{Power BI}
Power BI adalah platform \textit{business intelligence} dan visualisasi data yang dikembangkan oleh Microsoft untuk membantu pengguna mengubah data mentah menjadi informasi yang terstruktur dan mudah dipahami melalui laporan serta \textit{dashboard} interaktif. Platform ini terintegrasi erat dengan berbagai layanan Microsoft lainnya seperti Excel, SQL Server, dan Azure, sehingga proses pengambilan, pengolahan, dan pemuatan data dapat dilakukan dalam satu ekosistem yang saling terhubung. Komponen utamanya mencakup Power BI Desktop sebagai lingkungan perancangan model data dan laporan, Power BI Service sebagai layanan berbasis \textit{cloud} untuk publikasi serta kolaborasi, dan Power BI Mobile yang memungkinkan pengguna memantau laporan dan indikator kinerja dari perangkat seluler kapan saja dan di mana saja.

Sebagai alat analitik, Power BI dirancang untuk mempermudah proses pengumpulan, pengolahan, analisis, dan visualisasi data secara cepat dan relatif mudah dipelajari, sehingga pengguna dapat memperoleh wawasan yang relevan untuk mendukung pemantauan performa dan pengambilan keputusan berbasis data yang lebih efektif \cite{putraPelatihanPowerBI2023}. Berbagai jenis visualisasi seperti grafik, tabel interaktif, dan KPI dapat dikombinasikan dalam satu \textit{dashboard}, sementara integrasi layanan \textit{cloud} memungkinkan beberapa pengguna berkolaborasi pada laporan yang sama secara real time dan mengaksesnya dari berbagai perangkat dengan tetap mempertahankan konsistensi data.

\subsubsection{Google Looker Studio}
Looker Studio merupakan platform visualisasi data interaktif berbasis web yang memungkinkan pengguna mengintegrasikan data dari berbagai sumber dan menyajikannya dalam bentuk grafik, tabel, serta indikator yang mudah dipahami \cite{iskandarIMPLEMENTASIBUSSINESINTELEGENT2025}. Keberadaan konektor bawaan ke berbagai sumber data serta kemampuan menyusun \textit{dashboard} membuat Looker Studio banyak dimanfaatkan sebagai alat \textit{business intelligence} untuk menggabungkan data yang sebelumnya tersebar dan menyederhanakan informasi kompleks ke dalam satu tampilan analitis yang informatif. Antarmukanya yang intuitif dan berbasis \textit{drag-and-drop} juga dilaporkan memudahkan pengguna non-teknis dalam menyusun laporan visual secara cepat, sehingga proses analisis dan pelaporan menjadi lebih efisien.​

Dari sisi keunggulan, sejumlah studi menunjukkan bahwa pemanfaatan Looker Studio dapat meningkatkan efisiensi pengolahan data, mempercepat proses penyusunan laporan, serta mendukung pengambilan keputusan berbasis data melalui visualisasi yang aktual dan mudah diinterpretasikan. Implementasi \textit{dashboard} berbasis Looker Studio juga dilaporkan mempermudah kegiatan monitoring kinerja, meningkatkan kecepatan akses informasi, dan memberikan tampilan \textit{real-time} yang membantu identifikasi masalah maupun peluang secara lebih dini, termasuk dalam konteks pendidikan dan organisasi publik.
\section{\textit{Dashboard}}
\textcite{prasetiyaVisualisasiInformasiData2016} mendefinisikan \textit{dashboard} sebagai “tampilan visual dari informasi terpenting yang dibutuhkan untuk mencapai satu atau lebih tujuan, digabungkan dan diatur pada sebuah layar” sehingga pengguna dapat melihat informasi utama secara sekilas. \textit{Dashboard} biasanya berisi kombinasi teks dan grafik, dengan penekanan pada grafik, dan menampilkan informasi kritis institusi \cite{prasetiyaVisualisasiInformasiData2016}. Sebagai antarmuka satu layar penuh, \textit{dashboard} membantu pengambil keputusan cepat menangkap status kinerja terbaru.

Fungsi utama \textit{dashboard} adalah memudahkan pemantauan dan evaluasi kinerja organisasi secara berkesinambungan. Dalam pendidikan tinggi biasanya menunjukkan bahwa dashboard memadukan data dari berbagai sumber (misalnya sistem informasi akademik, hasil capaian, keuangan, SDM) dan menyajikannya secara terintegrasi untuk mendukung pengambilan keputusan.


\subsection{Dashboard dalam Monitoring dan Evaluasi di Perguruan Tinggi}
Monitoring dan evaluasi (monev) kinerja institusi adalah proses pengawasan dan penilaian berkelanjutan untuk memastikan tujuan pendidikan tinggi tercapai. Perguruan tinggi memerlukan alat monev yang handal untuk memastikan standar mutu nasional dan visi institusi terpenuhi (Permendiknas 19/2005). \textcite{hariyantiMODELPENGEMBANGANDASHBOARD2011a} menegaskan bahwa \textit{dashboard} merupakan salah satu alat yang tepat untuk melakukan monitoring dan evaluasi. Mereka menyatakan bahwa “dashboard merupakan alat yang digunakan untuk mengevaluasi proses yang sedang berjalan, memonitor kinerja yang sedang berjalan, serta untuk memprediksi kondisi di masa mendatang”. Dengan kata lain, \textit{dashboard} dapat dimanfaatkan untuk melihat capaian kinerja secara terus-menerus dan memberi sinyal dini bila terjadi penyimpangan dari target mutu.