% ==========================================
% BAB I PENDAHULUAN
% ==========================================
\chapter{PENDAHULUAN}
\label{chap:pendahuluan}
% --- Latar Belakang ---
\section{Latar Belakang}
Pemeringkatan perguruan tinggi global telah menjadi faktor penentu yang semakin signifikan dalam pengelolaan institusi pendidikan tinggi di seluruh dunia. Dalam era globalisasi dan meningkatnya persaingan antar perguruan tinggi secara internasional, posisi dalam pemeringkatan dunia tidak hanya mencerminkan reputasi akademik, tetapi juga menjadi instrumen strategis yang memengaruhi kebijakan institusi, alokasi pendanaan, serta daya tarik terhadap mahasiswa dan tenaga akademik berkualitas\cite{nguyenInfluenceGlobalUniversity2024}. QS World University Rankings merupakan salah satu sistem pemeringkatan yang paling berpengaruh dan diakui secara luas, dengan metodologi yang mengevaluasi universitas berdasarkan indikator-indikator kinerja utama seperti \textit{academic reputation} (30\%), \textit{citations per faculty} (20\%), \textit{employer reputation} (15\%), \textit{faculty-student ratio} (10\%), \textit{international research network} (5\%), \textit{international faculty ratio} (5\%), \textit{international student ratio} (5\%), \textit{employment outcomes} (5\%), dan \textit{sustainability} (5\%) \cite{QSWorldUniversity2025}.

Pengaruh pemeringkatan terhadap pengambilan keputusan institusional pendidikan telah terdokumentasi dengan baik dalam berbagai penelitian. Pemeringkatan global terbukti memengaruhi keputusan institusi dalam beberapa aspek, termasuk perencanaan strategis, rekrutmen dan pengembangan staf akademik, penjaminan mutu, alokasi sumber daya, serta penerimaan mahasiswa \cite{instituteforhighereducationpolicyImpactCollegeRankings2009}. Universitas tidak hanya menjadi penerima pasif dari pengaruh pemeringkatan, melainkan secara aktif mencari peluang untuk memanfaatkannya sebagai keunggulan strategis, terutama ketika pemeringkatan menguntungkan visibilitas institusi \cite{velizInfluenceGlobalRankings2022}.  \textcite{hazelkornRankingsReshapingHigher2015} dalam penelitiannya mencatat bahwa pemeringkatan telah menjadi pendorong dominan dalam pengambilan keputusan strategis universitas, dengan berhasil memfokuskan perhatian pada aspek kualitas meskipun dengan berbagai konsekuensi yang perlu dicermati.

Di Indonesia, perhatian terhadap pemeringkatan global semakin intensif seiring dengan upaya pemerintah dan perguruan tinggi untuk meningkatkan daya saing di kancah internasional. Berdasarkan data QS World University Rankings 2026, sebanyak 26 universitas Indonesia berhasil masuk dalam pemeringkatan, dengan Universitas Indonesia menempati posisi tertinggi di peringkat 189, diikuti oleh Universitas Gadjah Mada (peringkat 224), dan Institut Teknologi Bandung (peringkat 255), yang naik 25 peringkat dari tahun sebelumnya dengan skor 49,9 poin, menegaskan keberhasilan strategi institusional pendidikan dalam merespons kriteria pemeringkatan

Untuk mendukung visi jangka panjang institusi, rektor ITB secara strategis telah menetapkan target masuk ke peringkat 150 dunia QS pada tahun 2030. Target ini menjadi pendorong seluruh \textit{civitas academica} ITB untuk secara aktif meningkatkan performa di setiap indikator QS dan membangun budaya kerja yang berbasis pada data dan capaian. \textit{Dashboard} kinerja menjadi instrumen utama yang memungkinkan setiap kemajuan terhadap indikator QS dapat termonitor, sehingga setiap \textit{gap} dalam pencapaian dapat segera diidentifikasi dan direspons oleh unit terkait melalui perencanaan aksi yang terukur. \textit{Dashboard} tidak hanya menyediakan visualisasi, melainkan juga menjadi alat manajemen strategis yang mendorong keterlibatan kolektif \textit{civitas academica} dalam percepatan pencapaian tujuan institusi.

Monitoring dan evaluasi kinerja diperlukan untuk menjamin bahwa seluruh proses berjalan sesuai tujuan serta standar mutu yang telah ditetapkan \cite{hariyantiMODELPENGEMBANGANDASHBOARD2011a}. \textit{Dashboard} diperkenalkan sebagai alat visualisasi yang mampu menyajikan indikator kinerja utama secara ringkas dalam satu tampilan. Sebagai media yang mendukung pengambilan keputusan cepat, \textit{dashboard} dirancang untuk menghadirkan informasi penting dalam bentuk yang mudah dianalisis dan dipahami \cite{malikEnterpriseDashboardsDesign2005}. Integrasi data menjadi faktor kunci keberhasilan \textit{dashboard}, karena indikator kinerja memerlukan sumber data yang konsisten, terstruktur, dan terpusat. \textit{Dashboard} modern dibangun di atas infrastruktur \textit{business intelligence}, yang memungkinkan menyatukan data dan penyajian metrik utama secara visual dan interaktif \cite{goncalvesDevelopingIntegratedPerformance2023}.

Institut Teknologi Bandung (ITB), sebagai salah satu perguruan tinggi terunggul di Indonesia yang konsisten berada dalam pemeringkatan QS World University Rankings, memiliki cakupan kinerja yang luas dan kompleks. Indikator-indikator yang menjadi acuan pemeringkatan QS, mulai dari reputasi akademik, dampak penelitian melalui sitasi, reputasi di kalangan pemberi kerja, hingga kolaborasi penelitian internasional, menuntut pengelolaan data yang terintegrasi dan transparan. Data capaian tersebut telah dihimpun oleh Satuan Penjaminan Mutu (SPM) ITB dari berbagai unit kerja dan menjadi bagian penting dalam proses evaluasi mutu internal serta akreditasi institusi. Namun, informasi yang terkandung di dalamnya belum sepenuhnya transparan dan belum dapat diakses secara terbuka oleh \textit{civitas academica}. Ketiadaan media penyajian yang terpusat dan mudah dijangkau menyebabkan pemangku kepentingan institusi pendidikan seperti, rektor, kepala program studi, senat, dosen, tenaga kependidikan, maupun unit akademik kesulitan memperoleh informasi kinerja institusional secara utuh. Penelitian sebelumnya menunjukkan bahwa pengguna \textit{dashboard} di ITB menilai aspek penyajian data, personalisasi, dan performansi sistem sebagai fitur yang sangat penting untuk mendukung efektivitas monitoring kinerja \cite{hariyantiMODELPENGEMBANGANDASHBOARD2011a}.

\textit{Dashboard} kinerja memberikan cara yang lebih efektif dalam menyajikan informasi institusional melalui elemen visual seperti grafik, indikator warna, dan rangkuman metrik yang mudah dipahami oleh berbagai pemangku kepentingan. \textit{Dashboard} dirancang untuk menampilkan indikator kinerja utama dalam format visual yang ringkas, jelas, dan mudah dievaluasi oleh pengambil keputusan \cite{malikEnterpriseDashboardsDesign2005}. Dalam konteks respons terhadap pemeringkatan global, \textit{dashboard} memungkinkan universitas untuk memantau secara kontinu pencapaian terhadap indikator-indikator QS, seperti jumlah publikasi dan sitasi per fakultas, proporsi mahasiswa dan staf internasional, serta skor kolaborasi penelitian internasional. Visualisasi yang komprehensif membantu menyederhanakan data akademik dan operasional yang jumlahnya besar, sehingga pola dan tren dapat terlihat secara lebih cepat dibandingkan membaca laporan numerik atau tabel panjang. Selain menampilkan data secara ringkas, \textit{dashboard} modern juga menawarkan interaktivitas yang memungkinkan pengguna melakukan eksplorasi data lebih dalam, menerapkan filter khusus, serta memeriksa indikator berdasarkan unit, periode, atau kategori tertentu \cite{goncalvesDevelopingIntegratedPerformance2023}. Kemampuan ini menjadikan \textit{dashboard} sebagai alat yang tidak hanya menampilkan informasi, tetapi juga mendukung analitik tingkat lanjut yang penting dalam proses monitoring institusional sehari-hari.

Dengan karakteristik tersebut, \textit{dashboard} berperan penting dalam membantu institusi memahami dinamika kinerja. Penyajian data yang intuitif memudahkan pimpinan maupun unit terkait untuk menilai perkembangan indikator, mengidentifikasi area yang memerlukan perhatian, dan melihat ketercapaian target berdasarkan data aktual yang terstruktur. Proses ini membuat pemantauan kinerja menjadi lebih efisien dan mengurangi ketergantungan pada laporan manual yang memakan banyak waktu. Dalam konteks pengelolaan perguruan tinggi, kemampuan \textit{dashboard} untuk mengonsolidasikan berbagai indikator pendidikan, penelitian, dan pengabdian dalam satu tampilan terintegrasi menjadi sangat relevan mengingat tingginya kompleksitas data di lingkungan akademik. Dengan demikian, \textit{dashboard} berfungsi sebagai wadah informasi yang membantu institusi menjalankan proses monitoring secara lebih terukur tanpa menggantikan peran analisis manajerial yang tetap diperlukan untuk interpretasi mendalam \cite{mitchellDevelopingUsingDashboard2013}.
% --- Rumusan Masalah ---
\section{Rumusan Masalah}
Pengelolaan data capaian kinerja yang berkaitan dengan indikator pemeringkatan QS World University Rankings di Institut Teknologi Bandung (ITB) saat ini telah dilakukan oleh berbagai unit, termasuk Satuan Penjaminan Mutu (SPM) dan unit pendukung lainnya, yang menghimpun data publikasi, sitasi, kolaborasi internasional, reputasi akademik, serta indikator penunjang lainnya. Namun, data tersebut belum tersaji dalam sebuah sistem terpadu yang secara khusus dirancang untuk memonitor dan mengevaluasi capaian indikator QS, sehingga informasi strategis yang dibutuhkan pimpinan dan pemangku kepentingan untuk mengejar target peringkat 150 dunia pada tahun 2030 belum sepenuhnya mudah diakses, dianalisis, dan dimanfaatkan. Kondisi ini berpotensi menghambat proses pengambilan keputusan berbasis data, terutama ketika diperlukan respons cepat terhadap dinamika kinerja indikator pemeringkatan global.​

Selain itu, belum tersedianya \textit{dashboard} kinerja institusional yang secara eksplisit memetakan dan memvisualisasikan capaian tiap indikator QS seperti, \textit{academic reputation}, \textit{citations per faculty}, \textit{international research network}, maupun komposisi mahasiswa dan staf internasional membuat pemangku kepentingan harus meninjau data dari berbagai sumber dan laporan statis yang terpisah. Hal ini menyulitkan identifikasi tren capaian, \textit{gap} terhadap target tahunan, serta prioritas aksi perbaikan pada level unit dan institusi. Tanpa sebuah dashboard QS \textit{oriented} yang terintegrasi dan mudah digunakan, upaya ITB untuk mengawal kemajuan menuju peringkat 150 dunia menjadi kurang terstruktur dan berisiko tidak optimal.

Berdasarkan kondisi tersebut, rumusan masalah dalam tugas akhir ini adalah sebagai berikut.
\begin{enumerate}
\item Bagaimana merancang dan mengembangkan sistem visualisasi dashboard yang mampu mengintegrasikan berbagai sumber data capaian indikator QS World University Rankings di tingkat institusi ITB?

\item Bagaimana bentuk visualisasi dan fitur interaktif pada dashboard yang paling efektif untuk menyajikan informasi capaian indikator QS agar mudah dipahami dan dimanfaatkan oleh pemangku kepentingan (pimpinan universitas, pengelola program studi, dan unit pendukung) dalam proses monitoring, evaluasi, dan pengambilan keputusan berbasis data?
\end{enumerate}

% --- Tujuan ---
\section{Tujuan}
Tugas akhir ini bertujuan untuk merancang dan membangun prototipe \textit{dashboard} visualisasi capaian kinerja indikator QS World University Rankings yang dapat membantu Institut Teknologi Bandung (ITB) dalam memantau kemajuan menuju target peringkat 150 dunia pada tahun 2030 secara lebih transparan, terstruktur, dan mudah diakses oleh pemangku kepentingan di tingkat universitas. \textit{Dashboard} ini diharapkan menjadi media penyajian data yang terbuka dan informatif, sehingga proses pemantauan, evaluasi, dan pengambilan keputusan berbasis data terkait indikator QS dapat dilakukan secara lebih efektif.

Secara khusus, tujuan dari tugas akhir ini adalah sebagai berikut.
\begin{enumerate}
\item	Merancang arsitektur sistem \textit{dashboard} yang mampu mengintegrasikan berbagai sumber data capaian indikator QS World University Rankings di ITB.
\item	Membangun prototipe \textit{dashboard} interaktif yang dapat menyajikan indikator capaian kinerja QS dalam bentuk visualisasi yang informatif dan mudah dipahami oleh pemangku kepentingan di tingkat universitas.
\end{enumerate}

% --- Batasan Masalah ---
\section{Batasan Masalah}
Untuk menjaga ruang lingkup tugas akhir tetap fokus dan dapat diselesaikan secara efektif dalam waktu serta sumber daya yang tersedia, maka batasan masalah dalam tugas akhir ini adalah sebagai berikut:
\begin{enumerate}
\item	Tugas akhir hanya mencakup perancangan dan pembangunan prototipe \textit{dashboard} capaian indikator kinerja yang relevan dengan QS World University Rankings di lingkungan Institut Teknologi Bandung (ITB), tanpa melakukan perluasan analisis atau implementasi ke universitas atau institusi lain.
\item	Data yang digunakan dalam tugas akhir ini berupa data sekunder yang bersumber dari dokumen internal ITB (misalnya laporan SPM dan unit terkait) serta sumber publik yang relevan dengan indikator QS. Tugas akhir ini tidak menggunakan data sensitif institusional dan tidak melibatkan pengumpulan data primer secara langsung dari responden.
\end{enumerate}

% --- Metodologi Pengerjaan TA ---
\section{Metodologi}
\label{chap:metodologi}
Metodologi yang digunakan dalam tugas akhir ini mengacu pada pendekatan \textit{Cross-Industry Standard Process for Data Mining} (CRISP-DM), yang umum digunakan dalam pengembangan sistem berbasis data analitik dan visualisasi. Selain itu, metodologi CRISP-DM adalah pendekatan yang fleksibel, sehingga pendekatan ini cocok digunakan dalam pengerjaan Tugas Akhir ini \cite{sulianta2023basic}. Pendekatan ini terdiri dari enam tahapan utama yang bersifat iteratif dan fleksibel. Fase-fase tersebut dapat dilihat pada \ref{gambar:fase-crisp-dm}, yang menggambarkan penerapan CRISP-DM dalam konteks pengembangan dashboard capaian kinerja di ITB.
\begin{figure}[H]1
    \centering
    \includegraphics[width=0.7\textwidth]{image/FaseCRISPDM.png}
    \caption{Fase Metodologi CRISP-DM}
    \label{gambar:fase-crisp-dm}
\end{figure}
\begin{enumerate}
\item	\textit{Business Understanding}

Tahap ini bertujuan untuk memahami konteks organisasi, target strategis institusi, serta kebutuhan pemangku kepentingan terhadap informasi capaian indikator QS World University Rankings di ITB. Aktivitas meliputi pemetaan peran unit-unit terkait seperti SPM ITB dan unit pengelola data kinerja, serta identifikasi kebutuhan transparansi dan monitoring indikator QS bagi pimpinan, pengelola program studi, dosen, tenaga kependidikan, dan pihak terkait lainnya.
\item	\textit{Data Understanding}

Tahap ini mencakup proses pengumpulan dan eksplorasi awal data capaian kinerja ITB yang relevan dengan indikator QS, seperti publikasi dan sitasi, proporsi dosen dan mahasiswa internasional, serta data pendukung reputasi dan kolaborasi. Data diperoleh dari sumber internal (misalnya SPM ITB atau unit pengelola kinerja) yang tersedia. Analisis awal dilakukan untuk memahami struktur, format, kelengkapan, dan keterbatasan data pada masing-masing indikator QS.
\item	\textit{Data Preparation}

Pada tahap ini dilakukan pembersihan, standarisasi, dan integrasi data dari berbagai sumber agar siap digunakan dalam proses visualisasi \textit{dashboard}. Langkah ini mencakup penyesuaian format antar sistem, penanganan nilai hilang atau tidak konsisten, pengelompokan data berdasarkan periode dan unit, serta penyusunan dataset terstruktur yang merepresentasikan capaian indikator QS sesuai kebutuhan analisis di tingkat institusi.
\item	\textit{Modeling}

Tahap ini berfokus pada perancangan model visualisasi dan struktur \textit{dashboard} yang memetakan indikator QS ke dalam tampilan yang ringkas dan informatif. Desain dashboard disusun berdasarkan kebutuhan pengguna (stakeholder ITB) dengan memperhatikan prinsip data \textit{storytelling}, \textit{usability}, dan \textit{information hierarchy}, sehingga pengguna dapat dengan mudah menelusuri capaian per indikator, tren waktu, serta gap terhadap target. Perangkat \textit{tools} visualisasi seperti Power BI, Looker Studio, atau Tableau digunakan untuk mengimplementasikan rancangan tampilan dan interaktivitas \textit{dashboard}.
\item	\textit{Evaluation}

Evaluasi dilakukan untuk memastikan \textit{dashboard} yang dikembangkan telah selaras dengan tujuan tugas akhir dan kebutuhan pemangku kepentingan dalam memantau kemajuan menuju target peringkat QS. Pengujian mencakup validasi kesesuaian indikator yang ditampilkan, konsistensi dan akurasi data, kejelasan tampilan visual, serta kemudahan interpretasi informasi oleh pengguna akhir melalui uji coba terbatas dan pengumpulan umpan balik.
\item	\textit{Deployment}

Tahap akhir adalah penyusunan dan penyajian prototipe \textit{dashboard} dalam lingkungan uji yang dapat diakses secara terbatas oleh pihak universitas untuk simulasi penggunaan. Tahap ini mencakup konfigurasi akses serta rekomendasi pengembangan lanjutan agar prototipe \textit{dashboard} dapat diintegrasikan lebih lanjut ke dalam ekosistem sistem informasi ITB di masa mendatang.
\end{enumerate}