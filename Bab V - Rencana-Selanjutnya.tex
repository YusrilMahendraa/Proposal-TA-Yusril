% ==========================================
% BAB V RENCANA SELANJUTNYA
% ==========================================
\chapter{RENCANA SELANJUTNYA}
\label{chap:rencana-selanjutnya}
\section{Rencana Implementasi}
Rencana implementasi untuk tugas akhir ini melibatkan beberapa langkah kunci yang harus dilakukan untuk memastikan keberhasilan penerapan sistem, yaitu terkait perkakas yang akan digunakan dan linimasa penyelesaian kedepannya. 
\subsection{Perkakas dan Teknologi yang Dibutuhkan}
Berikut pada tabel \ref{tbl:perkakas-teknologi} adalah beberapa perkakas dan teknologi yang akan digunakan dalam implementasi tugas akhir ini.
\begin{longtable}{|p{1cm}|p{4cm}|p{9cm}|}
\caption{Daftar Perkakas dan Deskripsi dalam Pengembangan Dashboard}
\label{tbl:perkakas-teknologi} \\
\hline
\textbf{No} & \textbf{Perkakas} & \textbf{Deskripsi} \\
\hline
\endfirsthead

\hline
\textbf{No} & \textbf{Perkakas} & \textbf{Deskripsi} \\
\hline
\endhead

\hline
\multicolumn{3}{r}{\textit{Bersambung ke halaman berikutnya}} \\
\endfoot

\hline
\endlastfoot

1 & Power BI Desktop 
  & Aplikasi untuk pengembangan model data, pembuatan visualisasi, dan desain dashboard \\
\hline

2 & Power Query 
  & Tool ETL untuk pembersihan, transformasi, dan integrasi data dari berbagai sumber \\
\hline

3 & Power BI Service 
  & Layanan cloud untuk publikasi dan distribusi dashboard kepada pengguna \\
\hline

4 & Microsoft Excel 
  & Aplikasi untuk penyimpanan dan validasi data sumber \\
\hline

5 & OneDrive / SharePoint 
  & Layanan penyimpanan cloud untuk menyimpan file data dan memfasilitasi refresh otomatis \\
\hline

6 & DAX (Data Analysis Expression) 
  & Bahasa formula untuk mendefinisikan measure dan logika perhitungan indikator \\
\hline

\end{longtable}
\subsection{Linimasa Implementasi}
Berikut pada tabel \ref{tbl:linimasa-implementasi} adalah linimasa rencana implementasi tugas akhir secara umumnya.
\begin{longtable}{|p{4cm}|p{10cm}|}
\caption{Timeline Pelaksanaan Tugas Akhir}
\label{tbl:linimasa-implementasi} \\
\hline
\textbf{Periode} & \textbf{Aktivitas} \\
\hline
\endfirsthead

\hline
\textbf{Periode} & \textbf{Aktivitas} \\
\hline
\endhead

\hline
\multicolumn{2}{r}{\textit{Bersambung ke halaman berikutnya}} \\
\endfoot

\hline
\endlastfoot

September -- Oktober 2024 
& Identifikasi masalah dan kebutuhan tugas akhir, pengumpulan studi literatur serta penentuan metode tugas akhir\\
\hline

November -- Desember 2024
& Persiapan dataset melalui koordinasi dengan SPM ITB, persiapan tools dan teknologi yang akan digunakan, konfigurasi environment pengembangan, serta penyusunan kerangka laporan \\
\hline

Januari -- Februari 2025
& Pelaksanaan tahap Data Understanding dan Data Preparation dengan analisis karakteristik data, pembersihan dan transformasi data menggunakan Power Query, serta penggabungan data dari berbagai sumber \\
\hline

Maret 2025
& Pelaksanaan tahap Modelling dan Visualization dengan perancangan model data, pembentukan relasi antar tabel, pendefinisian measure, serta pengembangan dashboard dengan berbagai tipe visualisasi \\
\hline

April 2025
& Pelaksanaan tahap Evaluation dengan pengujian akurasi data, validasi terhadap sumber asli, pengujian interaktivitas dashboard, pengumpulan feedback stakeholder, dan refinement berdasarkan masukan \\
\hline

Mei 2025
& Pelaksanaan tahap Deployment dengan publikasi dashboard ke Power BI Service, konfigurasi security dan access control\\
\hline

Juni 2025
& Penyelesaian laporan akhir, finalisasi penulisan semua bab, review dan revisi berdasarkan masukan pembimbing, serta persiapan untuk ujian sidang tugas akhir \\
\hline

\end{longtable}
\section{Rencana Anggaran dan Biaya}
Untuk mendukung pelaksanaan tugas akhir ini, berikut adalah rincian anggaran dan biaya yang diperlukan sesuai dengan yang dituliskan pada tabel \ref{tbl:rincian-anggaran-biaya}.
\begin{table}[h!]
\centering
\begin{tabular}{|c|l|p{7cm}|c|}
\hline
\textbf{No} & \textbf{Komponen} & \textbf{Deskripsi} & \textbf{Biaya (IDR)} \\
\hline
1 & Power BI Pro & Power BI Microsoft yang digunakan untuk membuat \textit{dashboard}, berbagi \textit{dashboard}, dan kolaborasi data . & 252.000 \\
\hline
\end{tabular}
\caption{Rincian Anggaran dan Biaya Tugas Akhir}
\label{tbl:rincian-anggaran-biaya}
\end{table}

\section{Desain Pengujian dan Evaluasi}
Tahap pengujian dan evaluasi pada tugas akhir ini sudah dijabarkan pada bab \ref{chap:desain-pengujian-evaluasi}.
\section{Analisis Risiko dan Mitigasi}
Berikut pada tabel \ref{tbl:analisis-risiko-mitigasi} adalah beberapa analisis risiko yang mungkin dihadapi selama pelaksanaan tugas akhir ini beserta strategi mitigasinya.
\begin{longtable}{|p{1cm}|p{4cm}|p{4cm}|p{4cm}|}
\caption{Daftar Risiko, Dampak, dan Strategi Mitigasi} \\
\hline
\textbf{No} & \textbf{Risiko} & \textbf{Dampak} & \textbf{Strategi Mitigasi}
\label{tbl:analisis-risiko-mitigasi} \\
\hline
\endfirsthead

\hline
\textbf{No} & \textbf{Risiko} & \textbf{Dampak} & \textbf{Strategi Mitigasi} \\
\hline
\endhead

\hline
\multicolumn{4}{r}{\textit{Bersambung ke halaman berikutnya}} \\
\endfoot

\hline
\endlastfoot

1 & Keterlambatan pengumpulan data dari SPM 
  & Pengembangan tertunda 
  & Komunikasi awal dengan SPM, menyiapkan data alternatif \\
\hline

2 & Inkonsistensi format dan kualitas data 
  & Hasil visualisasi tidak akurat 
  & Validasi data menyeluruh, penetapan standar format bersama SPM \\
\hline

3 & Perubahan kebutuhan dari stakeholder 
  & Revisi desain dashboard 
  & Review berkala dengan stakeholder, dokumentasi perubahan \\
\hline

4 & Performa dashboard lambat dengan volume data besar 
  & Dashboard tidak responsif 
  & Optimisasi model data, performance tuning, load testing \\
\hline

5 & Visualisasi data sulit dipahami oleh pengguna 
  & Dashboard tidak optimal digunakan untuk mendukung pengambilan keputusan 
  & Melibatkan pengguna dalam pengujian sistem dan memberikan pelatihan \\
\hline

6 & Isu keamanan dan akses data 
  & Data dapat diakses oleh pihak tidak berwenang 
  & Role-based access control, audit security, compliance dengan kebijakan ITB \\
\hline

\end{longtable}